% Elsevier defaults.
\documentclass[preprint,12pt]{elsarticle}

% Packages used.
\usepackage{lineno} % Line numbers for Elsevier format.
\usepackage{graphicx} % Use of images.
\usepackage[utf8]{inputenc} % Use of non-ASCII characters.
\usepackage[hidelinks]{hyperref} % References with interactive links.

% Bibliography management.
\usepackage{natbib}
% Define bibliography style.
\bibliographystyle{plainnat}


\journal{Instituto Tecnologico de Tijuana}

\begin{document}

\begin{frontmatter}

\title{Virtual Reality}
\author{Salcedo Morales Jose Manuel}
\address{Tijuana, Baja California, Mexico}

\begin{abstract}
	This project consists of a virtual reality application for treating
	the phobia of solitude.
\end{abstract}

\begin{keyword}
VR, virtual-reality, software, hardware, academic, research, phobia, solitude.
\end{keyword}

\end{frontmatter}

% Elsevier format.
\linenumbers

\newpage
\tableofcontents

\newpage
\section{Introduction}

\newpage
\section{Project Overview}
	\subsection{Description of the problem}
	Phobias are not an easy thing to live with. Furthermore, phobia of
	solitude not only affects the person that has it, it also impacts
	everyone that must keep up with it. \cite{Monophobia}
	\subsection{Goals}
	\begin{itemize}
		\item Create a software app in Unity, with support for VR
			hardware. The options being:
			\subitem Oculus Rift
			\subitem Google Cardboard
	\end{itemize}
	\subsection{Hypotheses or assumptions}
	\subsection{Justification}
	Given the usage of virtual reality is still considred questionable
	as a means to treat or even fix real fears, with the help from
	\citet{VRJustification}, a viable usage is given for patients using
	virtual reality as a means for recovery:
	\newline ''It provides exposure in a way that patients feel safe. We
	can go to a location together, and the patient can tell me what they're
	feeling and what they’re thinking. Traditionally, psychologists have
	treated such conditions by helping patients imagine they are facing a
	fear, mentally creating a situation where they can address their
	anxieties. Virtual reality takes this a step further.''
	\newline This meaning, the usage of virtual reality for people with
	disorders means that they can confront their fears, without them
	actually being in real danger. So long as the patient does is not
	relaxed at the tought that it is not real, the testing is safer. Even
	still, if the patient is relaxed trough the test, that could mean the
	treatment is giving results.
\newpage
\section{Theoretical framework}
	\subsection{Historical framework}
	Considering that including many elements that take up the history
	of virtual reality will take up a lot of space (and time to write),
	in this section, only main points from \citet{VRHistory} shall be
	written:
	\begin{itemize}
		\item 1838 – Stereoscopic photos \& viewers.
			\subitem 1838 : The stereoscope (Charles Wheatstone)
			\subitem 1849 : The lenticular stereoscope (David Brewster)
			\subitem 1939 : The View-Master (William Gruber)
		\item 1929 – Link Trainer The First Flight Simulator
		\item 1950s – Morton Heilig's Sensorama
		\item 1960 – The first VR Head Mounted Display
		\item 1961 Headsight – First motion tracking HMD
		\item 1965 – The Ultimate display by Ivan Sutherland
		\item 1968 – Sword of Damocles
		\item 1969 – Artificial Reality
		\item 1991 – Virtuality Group Arcade Machines
		\item 1993 – SEGA announce new VR glasses
		\item 1995 – Nintendo Virtual Boy
	\end{itemize}
	\subsection{Conceptual framework}
		\subsubsection{Definitions}
		For many technologies that are presented, a proper definition
		is needed for them. With the definitions used from
		\citet*{AndroidOS}, \citet*{GoogleCardboard},
		\citet*{OculusRift}, \citet*{UnityEngine},
		\citet*{VirtualReality}, the required definitions are presented
		for some technologies and/or terms that will be used.
		\begin{description}
			\item[Android (operating system)]
				``Android is a software platform and 
				operating system for mobile devices, 
				based on the Linux kernel, and developed 
				by Google and later the Open Handset 
				Alliance. It allows developers to write 
				managed code in the Java language, 
				controlling the device via Google
				developed Java libraries.'' \cite{AndroidOS}
			\item[Google Cardboard]
				``A 3D virtual reality headset constructed of
				cardboard, introduced in 2015. Designed by
				Google and made by third parties, Cardboard
				holds an Android smartphone and uses the
				Cardboard app or a third-party app to display
				a stereoscopic view. The app is controlled by
				head movement and the smartphone's built-in
				accelerometer, as well as a magnet slider on
				the unit that interacts with the phone's
				magnetometer.'' \cite{GoogleCardboard}
			\item[Oculus Rift]
				``The device 
				is a lightweight virtual reality headset that
				blocks your view of your surroundings and fully
				immerses you in a virtual world. The Rift lets
				you step into a game, look around in any
				direction and see the game environment all
				around you rather than on a flat screen
				surrounded by your living room decor. And you
				see it in 3D.'' \cite{OculusRift}
			\item[Unity (game engine)]
				``Unity (commonly known as Unity3D) is a game
				engine and integrated development environment
				(IDE) for creating interactive media, typically
				video games. As CEO David Helgason put it,
				Unity ``is a toolset used to build games, and
				it’s the technology that executes the graphics,
				the audio, the physics, the interactions,
				[and] the networking.'''' \cite{UnityEngine}
			\item[Virtual Reality]
				``Real-time interactive  graphics
				with three-dimensional models, combined with a
				display technology that gives the user the
				immersion in the model world and direct
				manipulation.'' \cite{VirtualReality}
		\end{description}
		\subsubsection{Legal framework}
		Legal issues can always arrive from usage of any technology.
		\newline However, common usage of virtual reality has raised
		suspisicion of legal use. With the help of \citet{VRLegal},
		some key points can be defined.

		\begin{itemize}
			\item
			``Some of the key legal issues that these stakeholders,
			along with brands and other advertisers sponsoring and
			providing VR programs and campaigns ("Brands"), should
			consider relate to intellectual property rights, such
			as trademark and copyright ("IP"), and right of
			publicity.  Generally, these legal issues are the same
			across the virtual and real worlds, but VR creates
			interesting twists in how the existing laws may
			apply.''
			\newline Generally, main issues arise from the
			use of trademarks and intellectual rights. Money spent
			creating and keeping these creations profitable are a
			key point.

			\item
			``When VR stakeholders import or incorporate music,
			photographs, names or likenesses of people, or brand
			names or logos into a virtual experience, the
			traditional laws of trademark, copyright, and right of
			publicity apply. This means that such use may require
			permission from the owners of the applicable rights.
			Whether permission is required depends on the nature of
			the use and the stakeholders involved.''
			\newline Either in a virtual world or a the real one,
			usage of intellectual properties should be taken care
			of, this being paying for the usage of them if a
			commercial gain is expected or keeping the usage of
			content free.

			\item
			``When it comes to the creation of content in the real
			world, whoever creates content owns it, unless certain
			narrow exceptions apply or ownership is changed by
			contract. However, the default ownership is not so
			clear in VR, where Users direct creation but the
			underlying code enabling the creation of images,
			virtual property, or other content was created or
			otherwise is controlled by the Platform. Thus, as
			between the User and the Platform, ownership may be
			disputed.''
			\newline Also bearing in mind intellectual property,
			an entity should consider the platforms where that
			creation will be put to sale/usage.
		\end{itemize}
		\subsubsection{Objective and benefits}
		\subsubsection{Tipology}
		\subsubsection{Theoretical bases}
	\subsection{Referential framework}
\newpage
\section{Methodology}
	\subsection{Population or universe / sample}
	\subsection{Type of study}
	\subsection{Description of the instrument}
	\subsection{Collection procedure}
	\subsection{Statistical information management procedure}

\newpage
\section{Results obtained and discussion}

\newpage
\section{Conclusions}

\newpage
\section{References}
\bibliography{References.bib}

\end{document}
