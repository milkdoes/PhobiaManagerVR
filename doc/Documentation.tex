% Elsevier defaults.
\documentclass[preprint,12pt]{elsarticle}

% Packages used.
\usepackage{lineno} % Line numbers for Elsevier format.
\usepackage{graphicx} % Use of images.
\usepackage[utf8]{inputenc} % Use of non-ASCII characters.
\usepackage[hidelinks]{hyperref} % References with interactive links.

% Bibliography management.
\usepackage{natbib}
% Define bibliography style.
\bibliographystyle{plainnat}


\journal{Instituto Tecnologico de Tijuana}

\begin{document}

\begin{frontmatter}

\title{Virtual Reality}
\author{Salcedo Morales Jose Manuel}
\address{Tijuana, Baja California, Mexico}

\begin{abstract}
	This project consists of a virtual reality application for treating
	the phobia of solitude.
\end{abstract}

\begin{keyword}
VR, virtual-reality, software, hardware, academic, research, phobia, solitude.
\end{keyword}

\end{frontmatter}

% Elsevier format.
\linenumbers

\newpage
\tableofcontents

\newpage
\section{Introduction}

\newpage
\section{Project Overview}
	\subsection{Description of the problem}
	Phobias are not an easy thing to live with. Furthermore, phobia of
	solitude not only affects the person that has it, it also impacts
	everyone that must keep up with it. \cite{Monophobia}
	\subsection{Goals}
	\begin{itemize}
		\item Create a software app in Unity, with support for VR
			hardware. The options being:
			\subitem Oculus Rift
			\subitem Google Cardboard
	\end{itemize}
	\subsection{Hypotheses or assumptions}
	\subsection{Justification}

\newpage
\section{Theoretical framework}
	\subsection{Historical framework}
	\subsection{Conceptual framework}
		\subsubsection{Definitions}
		For many technologies that are presented, a proper definition is
		needed for them. With the definitions used from
		\citet*{AndroidOS}, \citet*{GoogleCardboard},
		\citet*{OculusRift}, \citet*{UnityEngine},
		\citet*{VirtualReality}, the required definitions are presented
		for some technologies and/or terms that will be used.
		\begin{description}
			\item[Android (operating system)]
				``Android is a software platform and 
				operating system for mobile devices, 
				based on the Linux kernel, and developed 
				by Google and later the Open Handset 
				Alliance. It allows developers to write 
				managed code in the Java language, 
				controlling the device via Google
				developed Java libraries.'' \cite{AndroidOS}
			\item[Google Cardboard]
				``A 3D virtual reality headset constructed of
				cardboard, introduced in 2015. Designed by
				Google and made by third parties, Cardboard
				holds an Android smartphone and uses the
				Cardboard app or a third-party app to display
				a stereoscopic view. The app is controlled by
				head movement and the smartphone's built-in
				accelerometer, as well as a magnet slider on
				the unit that interacts with the phone's
				magnetometer.'' \cite{GoogleCardboard}
			\item[Oculus Rift]
				``The device 
				is a lightweight virtual reality headset that
				blocks your view of your surroundings and fully
				immerses you in a virtual world. The Rift lets
				you step into a game, look around in any
				direction and see the game environment all
				around you rather than on a flat screen
				surrounded by your living room decor. And you
				see it in 3D.'' \cite{OculusRift}
			\item[Unity (game engine)]
				``Unity (commonly known as Unity3D) is a game
				engine and integrated development environment
				(IDE) for creating interactive media, typically
				video games. As CEO David Helgason put it,
				Unity ``is a toolset used to build games, and
				it’s the technology that executes the graphics,
				the audio, the physics, the interactions,
				[and] the networking.'''' \cite{UnityEngine}
			\item[Virtual Reality]
				``Real-time interactive  graphics
				with three-dimensional models, combined with a
				display technology that gives the user the
				immersion in the model world and direct
				manipulation.'' \cite{VirtualReality}
		\end{description}
		\subsubsection{Legal framework}
		\subsubsection{Objective and benefits}
		\subsubsection{Tipology}
		\subsubsection{Theoretical bases}
	\subsection{Referential framework}
\newpage
\section{Methodology}
	\subsection{Population or universe / sample}
	\subsection{Type of study}
	\subsection{Description of the instrument}
	\subsection{Collection procedure}
	\subsection{Statistical information management procedure}

\newpage
\section{Results obtained and discussion}

\newpage
\section{Conclusions}

\newpage
\section{References}
\bibliography{References.bib}

\end{document}
