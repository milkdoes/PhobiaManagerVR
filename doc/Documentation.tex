% Elsevier defaults.
\documentclass[preprint,12pt]{elsarticle}

% Packages used.
\usepackage{lineno}
\usepackage{graphicx}
\usepackage[utf8]{inputenc}
\usepackage[hidelinks]{hyperref}

% Define bibliography style.
\usepackage{natbib}
\bibliographystyle{elsarticle-num}

\journal{Instituto Tecnologico de Tijuana}

\begin{document}

\begin{frontmatter}

\title{Virtual Reality}
\author{Salcedo Morales Jose Manuel}
\address{Tijuana, Baja California, Mexico}

\begin{abstract}
	This project consists of a virtual reality application for treating
	the phobia of solitude.
\end{abstract}

\begin{keyword}
VR, virtual-reality, software, hardware, academic, research, phobia, solitude.
\end{keyword}

\end{frontmatter}

% Elsevier format.
\linenumbers

\newpage
\tableofcontents

\newpage
\section{Introduction}

\newpage
\section{Project Overview}
	\subsection{Description of the problem}
	Phobias are not an easy thing to live with. Furthermore, phobia of
	solitude not only affects the person that has it, it also impacts
	everyone that must keep up with it. \cite{Monophobia}
	\subsection{Goals}
	\begin{itemize}
		\item Create a software app in Unity, with support for VR
			hardware. The options being:
			\subitem Oculus Rift
			\subitem Google Cardboard
	\end{itemize}
	\subsection{Hypotheses or assumptions}
	\subsection{Justification}

\newpage
\section{Theoretical framework}
	\subsection{Legal framework}
	\subsection{Objective and benefits}
\newpage
\section{Methodology}
	\subsection{Population or universe / sample}
	\subsection{Type of study}
	\subsection{Description of the instrument}
	\subsection{Collection procedure}
	\subsection{Statistical information management procedure}

\newpage
\section{Results obtained and discussion}

\newpage
\section{Conclusions}

\newpage
\section{References}
\bibliography{References.bib}

\end{document}
