% Elsevier defaults.
\documentclass[preprint,12pt]{elsarticle}

% Packages used.
\usepackage{amssymb}
\usepackage{lineno}
\usepackage{graphicx}
\usepackage[utf8]{inputenc}
\usepackage[hidelinks]{hyperref}
\usepackage{caption}
\usepackage{listings}
\usepackage{dirtytalk} % For quoting text.
\usepackage{tabularx} % For longer tables.
\usepackage{ltablex} % For tables that go to new pages.
\usepackage{tabu}

% Define bibliography style.
\usepackage{natbib}
\bibliographystyle{elsarticle-num}

\journal{Instituto Tecnologico de Tijuana}


\begin{document}

\begin{frontmatter}

\title{Virtual Reality}

\author{Salcedo Morales Jose Manuel}

\address{Mexico, Baja California, Tijuana}

\begin{abstract}
	This project consists of a virtual reality application for treating
	the phobia for solitude.
\end{abstract}

\begin{keyword}
VR, virtual-reality, software, hardware, academic, research, phobia, solitude.
\end{keyword}

\end{frontmatter}

% Elsevier format.
\linenumbers

\newpage
\tableofcontents

\newpage
\section{Introduction}

\newpage
\section{Project Overview}

	\subsection{Description of the problem}
	\subsection{Goals}
	\subsection{Hypotheses or assumptions}
	\subsection{Justification}

\newpage
\section{Theoretical framework}
	\subsection{Legal framework}
	\subsection{Objective and benefits}
\newpage
\section{Methodology}
	\subsection{Population or universe / sample}
	\subsection{Type of study}
	\subsection{Description of the instrument}
	\subsection{Collection procedure}
	\subsection{Statistical information management procedure}

\newpage
\section{Results obtained and discussion}

\newpage
\section{Conclusions}

\newpage
\section{References}
\bibliography{Bibliography}

\end{document}
